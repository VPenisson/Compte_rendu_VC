\PassOptionsToPackage{unicode=true}{hyperref} % options for packages loaded elsewhere
\PassOptionsToPackage{hyphens}{url}
%
\documentclass[]{article}
\usepackage{lmodern}
\usepackage{amssymb,amsmath}
\usepackage{ifxetex,ifluatex}
\usepackage{fixltx2e} % provides \textsubscript
\ifnum 0\ifxetex 1\fi\ifluatex 1\fi=0 % if pdftex
  \usepackage[T1]{fontenc}
  \usepackage[utf8]{inputenc}
  \usepackage{textcomp} % provides euro and other symbols
\else % if luatex or xelatex
  \usepackage{unicode-math}
  \defaultfontfeatures{Ligatures=TeX,Scale=MatchLowercase}
\fi
% use upquote if available, for straight quotes in verbatim environments
\IfFileExists{upquote.sty}{\usepackage{upquote}}{}
% use microtype if available
\IfFileExists{microtype.sty}{%
\usepackage[]{microtype}
\UseMicrotypeSet[protrusion]{basicmath} % disable protrusion for tt fonts
}{}
\IfFileExists{parskip.sty}{%
\usepackage{parskip}
}{% else
\setlength{\parindent}{0pt}
\setlength{\parskip}{6pt plus 2pt minus 1pt}
}
\usepackage{hyperref}
\hypersetup{
            pdftitle={Compte rendu sur la validation croisée},
            pdfauthor={Manon Mahéo - Valentin Penisson},
            pdfborder={0 0 0},
            breaklinks=true}
\urlstyle{same}  % don't use monospace font for urls
\usepackage[margin=1in]{geometry}
\usepackage{graphicx,grffile}
\makeatletter
\def\maxwidth{\ifdim\Gin@nat@width>\linewidth\linewidth\else\Gin@nat@width\fi}
\def\maxheight{\ifdim\Gin@nat@height>\textheight\textheight\else\Gin@nat@height\fi}
\makeatother
% Scale images if necessary, so that they will not overflow the page
% margins by default, and it is still possible to overwrite the defaults
% using explicit options in \includegraphics[width, height, ...]{}
\setkeys{Gin}{width=\maxwidth,height=\maxheight,keepaspectratio}
\setlength{\emergencystretch}{3em}  % prevent overfull lines
\providecommand{\tightlist}{%
  \setlength{\itemsep}{0pt}\setlength{\parskip}{0pt}}
\setcounter{secnumdepth}{0}
% Redefines (sub)paragraphs to behave more like sections
\ifx\paragraph\undefined\else
\let\oldparagraph\paragraph
\renewcommand{\paragraph}[1]{\oldparagraph{#1}\mbox{}}
\fi
\ifx\subparagraph\undefined\else
\let\oldsubparagraph\subparagraph
\renewcommand{\subparagraph}[1]{\oldsubparagraph{#1}\mbox{}}
\fi

% set default figure placement to htbp
\makeatletter
\def\fps@figure{htbp}
\makeatother


\title{Compte rendu sur la validation croisée}
\author{Manon Mahéo - Valentin Penisson}
\date{29/09/2021}

\begin{document}
\maketitle

\hypertarget{introduction}{%
\subsection{Introduction}\label{introduction}}

\emph{l'intérêt de ces approches (à quoi servent-elles ?) en
apprentissage statistique supervisé. N.B.: une attention particulière
devra être portée à la rigueur des objets et arguments mathématiques
invoqués}

La performance du modèl te ou algorithme issu d'une méthode d'appren-
tissage s'évalue par un risque ou erreur de prévision, dite encore
capacité de généralisation

La

propriétés du risque ou erreur de prévision ou er- reur de
généralisation dans le cas de la régression et de la clas- sification

On dispose d'un échantillon de données observées de type entrée-sortie
de taille n : \(d^n_1 = \left\{(x_1,y_1),...,(x_n,y_n)\right\}\) avec
\(x_i \in \cal{X}\) quelconque (souvent égal à \(\mathbb{R}^d\)),
\(y_i \in \cal{Y}\) pour \$i = 1\ldots{}n \$.

L'objectif est de prédire la sortie \(y\) associée à une nouvelle entrée
\(x\), sur la base de \(d^n_1\).

On suppose que \(d^n_1\) est l'observation d'un \(n\)-échantillon
\(D^n_1 = \left\{(X_1,Y_1),...,(X_n,Y_n)\right\}\) d'une loi conjointe
\(P\) sur \(\cal{X} x \cal{Y}\) , totalement inconnue. Mais aussi que x
est une observation de la variable X et que (X,Y) est un couple
aléatoire de loi conjointe P indépendant de \(D^n_1\).

Une \textbf{règle de prédiction (régression ou discrimination)} est une
fonction (mesurable) \(f : \cal{X} → \cal{Y}\) qui associe la sortie
\(f(x)\) à l'entrée \(x ∈ \cal{X}\).

Dans tous les cas, ces règles optimales dépendent de P ! 􏰣→ Nécessité de
construire des règles - ou algorithmes ou modèle - de prédiction qui ne
dépendent pas de P, mais de D1n (et de paramètres à ajuster).

\(\boxed{p = 2\pi r}\)

Pour estimer le risque moyen

\hypertarget{description-des-muxe9thodes}{%
\subsection{Description des
méthodes}\label{description-des-muxe9thodes}}

\emph{Pour chaque méthode : une fiche descriptive s'appuyant notamment
sur un ou des schémas inédits} ;* \#\#\# Validation (croisée) hold out

\begin{itemize}
\tightlist
\item
  Concept
\item
  Avantages
\item
  Inconvénients
\end{itemize}

\hypertarget{validation-croisuxe9e-leave-p-out}{%
\subsubsection{Validation croisée
leave-p-out}\label{validation-croisuxe9e-leave-p-out}}

\begin{itemize}
\tightlist
\item
  Concept
\item
  Avantages
\item
  Inconvénients
\end{itemize}

Faire une petite partie sur leave-one-out

\hypertarget{validation-croisuxe9e-k-fold}{%
\subsubsection{Validation croisée k
fold}\label{validation-croisuxe9e-k-fold}}

\begin{itemize}
\tightlist
\item
  Concept
\item
  Avantages
\item
  Inconvénients
\end{itemize}

\hypertarget{bootstrap}{%
\subsubsection{Bootstrap}\label{bootstrap}}

\hypertarget{conclusion}{%
\subsection{Conclusion}\label{conclusion}}

\emph{les liens éventuels entre les différentes méthodes et des
recommandations à destination des practicien.ne.s.}

\end{document}
